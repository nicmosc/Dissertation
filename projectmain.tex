\documentclass[12pt]{ociamthesis}  % default square logo 
%\documentclass[12pt,beltcrest]{ociamthesis} % use old belt crest logo
%|\documentclass[12pt,shieldcrest]{ociamthesis} % use older shield crest logo

%load any additional packages
\usepackage{amssymb}
\usepackage{url}
%\usepackage[toc,page]{appendix}
\usepackage{graphicx}
\graphicspath{{./figures/}}
\DeclareGraphicsExtensions{.pdf,.png}

%============ BIBLIOGRAPHY AND REFS PACKAGES ========%
%\usepackage{multibib}
%\newcites{x}{References}
%\newcites{sec}{Background Reading}

%============ CUSTOM CHAPTER NUMBER ========%

% ============== stuff below is for header\footer
\usepackage{fancyhdr}
\pagestyle{fancy}
\fancyhead{}
\renewcommand{\sectionmark}[1]{\markright{\textit{#1}}}
\renewcommand{\chaptermark}[1]{\markboth{\textsc{#1}}{}}
%\fancyhead[LO]{\leftmark}
%\fancyhead[RE]{\rightmark}
\fancyhead[R]{\rightmark}
\fancyhead[L]{\leftmark}
\cfoot{\fancyplain{}{\thepage}}

%\makeatletter
%\def\@makechapterhead#1{%
%  \vspace*{50\p@}%
%  {\parindent \z@ \raggedright \normalfont
%    \interlinepenalty\@M
%    \Huge\bfseries  \thechapter.\quad #1\par\nobreak
%    \vskip 40\p@
%  }}
%\makeatother

% ============== FRONT PAGE =========== %
        
\title{Automated Visualised Translation from English to British Sign Language}

\author{Nicolaos Moscholios}             %your name
\college{Pembroke College}  %your college

\degree{Master of Computer Science}     %the degree
\degreedate{Trinity 2016}         %the degree date

%end the preamble and start the document
\begin{document}

%this baselineskip gives sufficient line spacing for an examiner to easily
%markup the thesis with comments
\baselineskip=18pt plus1pt

\maketitle                  % create a title page from the preamble info

\begin{abstract}
A large number of people in the world today are born deaf and many rely on British Sign Language (BSL), the most widely used method of signed communication in the UK. BSL is structured in a completely different way to English and, like any language, it has its own grammar. In order to create a communication bridge between English speakers and individuals affected by deafness there have been attempts at building a formal model of translation. ...
\end{abstract}

\begin{acknowledgements}
I would like to thank...
\end{acknowledgements}

\begin{notes}
The work described in this thesis is available online at \url{nicmosc.com/bsltranslate}. Examples discussed throughout the report marked by an asterisk (*) can be viewed on the website. For demonstrative purposes it is highly suggested that the Examiner try those examples ..
\end{notes}

%\include{abstract}          % include the abstract
{\pagestyle{plain}
	\begin{romanpages}          % start roman page numbering
	%set the number of sectioning levels that get number and appear in the contents
	\setcounter{tocdepth}{5}
	\setcounter{secnumdepth}{5}
	\tableofcontents            % generate and include a table of contents
	\listoffigures              % generate and include a list of figures
	\end{romanpages}            % end roman page numbering
\cleardoublepage}

%===================== CHAPTER 1 ===================%

\chapter{Introduction}
For most people around the world, communication is achieved orally by speaking. Unfortunately some are born deaf or are affected by hearing impairment with time. It is then necessary to use another medium of communication. Today there are about 1 million "functionally deaf" UK individuals \citex{website:bsl-stats}, that is, that require the use of sign language to converse. Additionally there are around 11 million people affected by some degree of hearing loss and while many benefit from hearing aids, sign language bypasses the need to speak, making it easy for everyone to understand each other.  

\section{Motivation}
The problem arises when hearing and Deaf individuals need to communicate. Deaf individuals are taught to read and write English from a young age, but it was found that they have difficulties understanding text with a reading level above that of primary school students \citex{paper:deaf-edu}. Thus it is not enough for hearing individuals to rely on written text to communicate with their community. More and more people are learning BSL even though they have perfectly fine hearing. There are certificates that can be obtained to work with Deaf individuals \citex{website:signature}. Taking lessons from BSL teachers and practising with the community helps to learn the language.

Online resources such as UCL's BSL Sign Bank \citex{website:bsl-sign-bank} and SignBSL.com \citex{website:sign-bsl} offer clips of signers showing words and phrases in BSL for learners and curious people?. The main issue is that they are highly static. Analogously to an English dictionary, it is only useful for looking up individual signs and not \textit{translating} phrases. Automated translation has been an ongoing challenge in the field of
computational linguistics, and we are finally seeing very positive results thanks to the recent developments in machine learning. Since spoken (written) languages are heavily dependent on their syntactic components, they are relatively easy to formalise in such a way that a machine can interpret a sentence and perform operations on it to transition from one language to another. More recently, there have also been attempts at doing so with signed languages such as ASL (American Sign Language) and BSL. Given that there is no formally typed grammar for BSL, it is not easy to find a model that encapsulates all key elements involved, such as facial expressions and eye gaze. In addition, since the language is only signed, the only way to see if an interpretation is correct, is through visual output. While some past implementations have been successful, due to the then high computational power necessary to run these systems, most of them are now outdated or have been left at a prototype stage. In 2006, IBM created a fully working system \citex{website:IBM} that allowed users to speak and see the translation from English to BSL in real time; however it has been disregarded and is not available any more. 

\section{Objectives}
This project will build upon previous work and attempt to solve the interlingual translation from English to British Sign Language by implementing a web application where signs are visualised through a virtual avatar. Behaviour will closely follow that of other established translation systems such as Google Translate where users can type a sentence and get a result in real-time. Through rule-based translation methods (discussed in section \ref{machine translation}) and 3D animation techniques the system will allow users to not only find out what a particular sentence corresponds to in BSL, but also learn and hopefully practice their skills. Figure \ref{fig:sys-overview} shows a conceptual overview of the system pipeline; this figure will be references multiple times throughout the report.

\begin{figure}[h]
	\centering
    \includegraphics[scale=0.75]{system-overview}
    \caption{Conceptual overview of the system}
    \label{fig:sys-overview}
\end{figure}

\section{Structure}
The report will read as follows: in chapter 2 we will give a brief introduction to the linguistics British Sign Language and its differences from written English, as well as existing methods of translation from written to written and from written to signed languages. Chapter 3 will cover the design choices for the system, that is, the development environment with the programming languages, the conceptual structure of the pipeline and the end-user interaction. Following up in chapter 4 we explain in detail the implementation of both the translation methods and the animation techniques utilised, as well as the communication between the two modules. Finally chapter 5 will discuss the evaluation of the system both in terms of translation accuracy and feedback from the user survey, while chapter 6 concludes the report along with project achievements, personal remarks and future work.

%===================== CHAPTER 2 ===================%

\chapter{Background}
\section{British Sign Language: Linguistics Overview}
This section will give a short introduction to the linguistics of British Sign Language, including common grammar rules, its morphology and how it differs from written English. Please note that only the most general information is described here, that is enough to follow concepts described in section \ref{existing methods} and the implementation of the system (section \ref{implementation}).

British Sign Language, often shortened to BSL, has been an officially recognised language since 2003 \citex{website:signature}. Like English, it has its own grammar \citex{website:deafness} and its own lexicon. While not as large as its written counterpart, there are enough signs to convey the same ideas in sign language, as many can be combined (\textit{compounds}, see section \ref{linguistics}) to create new meanings. However BSL is not the only signed language; each country has their own with different dialects by region. ...  

Talk about non-manual features here

Since BSL can only be transmitted through signing, there is no formal written form. Many different formats exist such as Stokoe, designed only for representing hand movements (thus carries no information about non-manual features), HamNoSys (Hamburg Notation System) that can represent any signed language, and more. The most commonly used method to represent sign language without needing to learn the notation (both Stokoe and HamNoSys require additional knowledge to be decoded) is through \textit{gloss}. Gloss uses the most basic representation of the sign in its English written form. For example if one wants to describe a situation where a girl is eating an apple, the gloss would be GIRL EAT APPLE. 

\subsection{Morphology and Morphemes}
A morpheme in BSL is considered as the smallest unit of meaning in a word or sign. We can combine them to form signs that have several meaningful parts but are still considered a single sign. We distinguish between \textit{monomorphemic} signs (cannot be further subdivided) like TRUE, SAY, MOUSE and \textit{polymorphemic} which are a combination of 
\subsection{Verbs}
\subsubsection{Tense, Aspect and Manner}
\subsubsection{Verb types}
Topographic vs. Syntactic space, plain, agreement, spatial verbs

\subsection{Questions and Negations}
Wh-, Facial/Head movements etc.
\subsection{Fingerspelling}

\subsection{Non-manual features}
\subsubsection{Spoken and Oral Components}
Spoken components, oral components
\subsubsection{Facial Expressions and Head Movements}
\subsubsection{Eye Gaze}

\section{Online Resources}

\section{Machine Translation}
\label{machine translation}
\subsection{Classical}
\subsection{Statistical}

\section{Existing Methods}
\label{existing methods}
\section{Personal Work}
\section{Summary}

%===================== CHAPTER 3 ===================%

\chapter{Design}

\section{Language Choice}
\section{Translation}
\section{User Interface}
	\subsection{User Interaction}
\section{Summary}

%===================== CHAPTER 4 ===================%

\chapter{Implementation}
\label{implementation}

\section{Rule-Based MT in Python}
	\subsection{Pipeline}
	\subsection{System Components}
		\subsubsection{Grammar Tree Transforms}
		\subsubsection{Direct Translation}
		\subsubsection{Special Cases}
		\subsubsection{Output}
	
\section{Animation in ThreeJS}
	\subsection{Blender}
		\subsubsection{Model}
		\subsubsection{Skeletal Animation}
	\subsection{Data Format}
		\subsubsection{JSON Formatter}
	\subsection{Animation Engine}
		\subsubsection{Pipeline}
		\subsubsection{ThreeJS}
	
\section{Interlingual Communication with Flask}
	\subsection{Python to JavaScript}

\section{Summary}
			
%===================== CHAPTER 5 ===================%

\chapter{Evaluation}

\section{Translation}
	\subsection{BLEU}
\section{Animation}
\section{Survey}
\section{Feedback}

%===================== CHAPTER 6 ===================%

\chapter{Conclusion}

\section{Improvements}
	\subsection{Handling of Agreement Verbs}
	\subsection{Classifiers}
\section{Expansion}

%===================== APPENDIX ===================%

%now enable appendix numbering format and include any appendices
\appendix
\chapter{Extra code (appendix)}

%===================== REFREENCES ===================%
%next line adds the Bibliography to the contents page
\addcontentsline{toc}{chapter}{References}
\bibliographystylex{plainurl}
\bibliographyx{references.bib}
\newpage

\addcontentsline{toc}{chapter}{Background Reading}
\nocitesec{*}
\makeatletter
\renewcommand\@biblabel[1]{}
\makeatother
\bibliographystylesec{plainurl}
\bibliographysec{bibliography.bib}

\end{document}